
\documentclass{article}

% --- Package imports ---

\usepackage{
  amsmath, amsthm, amssymb, mathtools, dsfont,	  % Math typesetting
  graphicx, wrapfig, subfig, float,                  % Figures and graphics formatting
  listings, color, inconsolata, pythonhighlight,     % Code formatting
  fancyhdr, sectsty, hyperref, enumerate, enumitem ,algpseudocode} % Headers/footers, section fonts, links, lists

% --- Page layout settings ---

% Set page margins
\usepackage[left=1.35in, right=1.35in, bottom=1in, top=1.1in, headsep=0.2in]{geometry}

% Anchor footnotes to the bottom of the page
\usepackage[bottom]{footmisc}

% Set line spacing
\renewcommand{\baselinestretch}{1.2}

\title{UMC201: Binary Search And Formalisation}
\author{Sirjan Hansda}
\date{8th August 2023}


\begin{document}
\maketitle
\leavevmode
\\
\textcolor{red}{Question.} \textbf{We Optimise the Binary Search Code as Follows:} \\
\\
\textbf{INPUT:Array A, len(A)=n, target=x}
\begin{lstlisting}[mathescape=true]
$p=0,q=n-1$
while(p<q):
    $m=\lfloor \frac{p+q}{2} \rfloor$
    if x$\leq$A[m]:
        $q=m$
    else:
        $p=m+1$
if($x==A[p]$)
    return $p$
else:
    return($-1$)
\end{lstlisting}
\textbf{OPTPUT: -1 or position of x}
\\
\\
\textbf{(a)}Why does this code not work for this modification?
\begin{lstlisting}
if x<A[m]
    q=m
else:
    p=m+1

\end{lstlisting}
\leavevmode
\newline
\textbf{(b)Find the fix for the code and Prove that the fix works:}
\begin{lstlisting}[mathescape=true]
    $m=\lceil \frac{p+q}{2} \rceil $
    if ($x\leq A[m]$)
        $q=m$
    else:
        $p=m+1$
\end{lstlisting}

\newpage
\textcolor{blue}{Answer. (a)} The Code does not work for the given modification because, while making changes to $p$ and $q$
we are not checking for A[m] itself. Even if A[m]=x, then, we directly change p to m+1, thereby skipping over A[m]
\\
\\
\textbf{Example:} A=[1,3,5,10,15] and x=5. so, at first, p=0 and q=4, therfore m=2. Now, A[m]=x, but because of the modified logic, p is changed to m+1=3
.\\
Thus now, p=3 and q=4. Obviously, 5 is not present between them. Therfore, the given code fails to find the target value, even though it is in the array.
\\
\\
\textcolor{blue}{Answer. (b)} The code is not accurate. \\ To Prove this, we provide a counter-example:
\\
A=[1,3,7,11,25,34] and target=3. \\
So, in the first iteration: p=0, q=5. So, m=3.
A[3]=11 $\ge$ 3, so we change q=3. For the second iteration, p=0 and q=3. \\ Now, for second iteration m=2. Again, A[2] = 7 $\geq$ 3. So, we change, q=2. 
Thus for $3^{rd}$ iteration, p=0 and q=2.\\ Again, in this iteration, m=1. So, A[m]=3, but, by the given logic, we again set q=1. For the fourth iteration, p=0 and q=1.
\\In this iteration again we find m=1, and the same effect continues. Infact, this loop will never exit, thereby being unable to give the correct answer.
\leavevmode
\\
\\
\textbf{Proposed Fix:}
\begin{lstlisting}[mathescape=true]
p=0,q=n-1 
$\ldots$
$m=\lceil \frac{p+q}{2} \rceil$ #Unchanged
if ($x< A[m]$):
    q=m-1
else:
    p=m
\end{lstlisting}
\leavevmode
\\
\textbf{Proof that the new Algorithm works:}
\\
\\
\begin{center}
    \textbf{LOOP INVARIANT:} \\ (I)$A[0:p]$ $\leq$ $x$ OR p=0 \\
                                (II) $x<A[q+1:n-1]$
\end{center}
\textbf{INITIATION:} Initially, p=0 and q=n-1. So, $1^{st}$ invariant is satisfied. A[n:n-1] is empty array, and hence the $2^{nd}$ invariant statements is vacuously true.
\\
\\
\textbf{MAINTAINANCE:} We assume that the loop invariants are satisfied. Therefore, we enter the loop:
\\
\begin{center}
   (i) $0\leq p<q \leq n-1$ \quad \ldots from the loop condition and the initial values of p,q\\
   (ii) $A[0:p-1] \leq x $ \quad \ldots from $1^{st}$ invariant\\
   (iii) $A[q+1:n-1]>x $ \quad \ldots from $2^{nd}$ invariant\\ 
\end{center}
\leavevmode
Now, we compute $m=\lceil \frac{p+q}{2} \rceil $. From basic mathematics, we know that, $p<m\leq q$. Now, we divide into 2 cases:
\\
\\
1) $x<A[m]$: Then
\begin{center}
    $q'=m-1$\\
    $p'=p$\\
\end{center}
Since, $x<A[m]$ and because A is a sorted array, so,  $x<A[m:n-1] \Rightarrow x<A[q'+1:n-1]$. Thus, the second loop invariant holds true.
The first invariant also holds true, as the value of p remains same as the previous value. Thus, the invariants hold true.
\\
\\
2) $x \geq A[m]$: Then
\begin{center}
    $p'=m$ \\
    $q'=q$\\
\end{center}
Since, $x \geq A[m]$ and because A is a sorted array, so, $x\geq A[0:m] \Rightarrow x\geq A[0:p']$.Therefore, the $1^{st}$ invariant holds true. The second loop invariance also holds true, as no modification has been done to $q$.
\leavevmode
\\
\\
\textbf{TERMINATION:} When the loop terminates, $p=q$ and furthermore, going by the loop invariant, $A[0:p]\leq x < A[p+1:n-1]$. Thus, if $x \in A$ , then $x=A[p]$. Atlast, the algorithm checks if $A[p]=x$. If yes then we have found $x$ otherwise, we have failed to do so.
Thus, the invariance is maintained even after the loop exits. \textbf{Thus the algorithm is correct}.
\\
\\
\textbf{Proof that the given algorithm terminates:}
\\
The loop terminates only when p=q or p$>$q.
\\
\\
 What happens after every step is that, either $p'=\lceil \frac{p+q}{2} \rceil$ or $q'=\lceil \frac{p+q}{2} \rceil -1$. Thus,
\\
Initial Length=$q-p+1$\\
Final Length= $(\lceil \frac{p+q}{2} \rceil -1)-p+1= \lceil \frac{p+q}{2} \rceil-p$ \ldots if q is modified.
\\
Final Length= $q-(\lceil \frac{p+q}{2} \rceil)+1= q-\lceil \frac{p+q}{2} \rceil$ +1\ldots if p is modified.
\\
Now, we know,
\begin{center}
     $p<\lceil \frac{p+q}{2} \rceil \leq p$
\end{center}
So, no matter what happens, the length of the array always decreases. And we know, that only integers are acceptable as indices of arrays. \textbf{And since the list of integers have a least element, we are sure that at sometime, the length of the array must come down to 0.}
Thus the loop must terminate.

\end{document}